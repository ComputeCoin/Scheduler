\documentclass{article}

% Packages
\usepackage[utf8]{inputenc}
\usepackage[english]{babel}

\usepackage{amsthm}
\usepackage{amsfonts}
\usepackage{amssymb}
\usepackage{graphicx}
\usepackage{caption}
\usepackage{enumitem}
\usepackage{multirow}

\newenvironment{notation}
{
    \newpage
    
    \paragraph{\Huge Notation}
    \begin{flushleft}
}
{
    \end{flushleft}
}

\newcommand{\notate}[2]{$#1 \hspace{0.35in} \mbox{#2}$}

% Theorems
\theoremstyle{definition}

\newtheorem{definition}{Definition}[section]
\newtheorem{proposition}{Proposition}[section]

\theoremstyle{remark}

\newtheorem*{remark}{Remark}
\newtheorem*{theorem}{Theorem}

% Commands
\newcommand{\func}[3]{#1:#2\rightarrow#3}
\newcommand{\reqfunc}[2]{#1:#2\rightarrow\mathbb{B}}
\newcommand{\reqop}[2]{\mathbf{Req}_{#1}(#2)}

\newcommand{\path}[2]{\mathbf{Path}^{#2}_{#1}}
\newcommand{\pathall}[1]{\mathbf{Path}_{#1}}
\newcommand{\cyclepath}[1]{\mathbf{Cycle}_{#1}}

\newcommand{\labeledarrow}[1]{\stackrel{#1}{\rightarrow}}

\newcommand{\addpic}[1]{\includegraphics[width=1.0\textwidth]{#1}}

\newcommand{\setbuild}[3]
{
	\{\hspace{0.05in} 
	#1 \in #2 \hspace{0.05in} 
	| \hspace{0.05in}#3\hspace{0.05in}
	\}	
}

\makeatletter
\@addtoreset{section}{part}
\makeatother

% Preamble
\title{Scala Job Scheduler}
\date{March 2018}

% --------------------------------------------------------------------------
\begin{document}
	
	% Title page
	\pagenumbering{gobble}
	\maketitle
	
	% Table of contents
	\newpage
	\tableofcontents
	
	% Begin main document
	\newpage
	\pagenumbering{arabic}
	
	\addcontentsline{toc}{section}{Introduction}
	
	\section*{Introduction}

	\indent
	
	The aim of this project is to design an optimal scheduling algorithm for a scalable computing cloud, where
 	computing resources are dynamically allocated to meet the demands of an inhomogeneous set of consumers. Resources
 	are not uniformly distributed, geographically or otherwise, as the nodes comprising the cloud are of variable type
 	and processing power. Clients will submit job specifications (indicating the number and type of cores, ideal
	network topology, required run time, etc.) to the scheduler, which should designate a time to run and a cluster of
	nodes that adheres to the specification. The algorithm should maximize throughput (efficient use of resources)
	while maintaining fairness, with wait times minimal and consistent between jobs that place similar demands on the
	system.
	
	In order to build an algorithm capable of operating on a complex, heterogeneous system of resources and consumers,
	we must consider the process from an abstract yet granular point of view. In an attempt to do so we have defined
	the system in terms of simple mathematical objects, and constructed an algebra over those objects to describe
	their interaction. By decomposing the scheduling process into its constituent parts, we were able to describe
	each component of the process in terms of this algebra and thus devise a mathematical model for the process as a
	whole. 
	
	Due to the difficulty of processing such a dynamic, diverse system (given the sheer number of factors to account
	for), we elected to confront the problem with a reinforcement learning algorithm built on the aforementioned model.
	The algorithm is highly dependent on the system's partitioning into measurable (numerically describable)
	components, and typifying these components for efficient processing - thus the focus on separability and producing
	quantifiable descriptors of the system. What follows is a detailed specification of the resultant scheduling
	paradigm: first of the environment and scheduling process, component by component, and then the construction of
	the reinforcement model from those components.
	
\newpage
	
\part{Scheduling Scheme}
	
	\addpic{figures/Intro-General.jpg}

	\addpic{figures/all-process.jpg}
	
	\addpic{figures/Macro-fix.png} 
	
	
	Upon submission by user, a consumer (job) enters the scheduling process with its specifications. It leaves the scheduling process by starting consuming (running on) assigned resources. Once finished, the consumer gives feedback to the scheduler for further training improvement.
	
	Example: a job is submitted with specifications such as arrival time, maximum run time, required number of nodes or cores, memory size, CPU/GPU, etc. After it finishes running, it gives feedback, such as wait time, actual run time, user reward, etc, to the scheduler.
	
	\addpic{figures/Intro-Scheduler.jpg}
	
	The scheduler first takes in the consumer specifications and inspects the current resource space. Then both the specifications and resource states are collected by the training module. During the reinforcement training, the training module and resource space are constantly being updated, while resources are being assigned or released and consumers entering or leaving the scheduling process.
	

	\section{Specifications}
	
	\section{Consumer and Resource}
	
	\section{Main Process}
	
	\begin{enumerate}[label=\alph*.]
	
	\item Main Training Process
	
	\addpic{figures/Training-Main-Process.jpg}
	
	\item Neural Network
	
	\addpic{figures/Neural-Network.jpeg}
	
	\item Policy Generator

	\addpic{figures/Policy-Generator.png}		
	
	Bayesian Hierarchical Clustering
	
	\item Bins
	
	\item Joint Actions

	\end{enumerate}

	\addpic{figures/Joint-action.jpg}

	
	
	At the end of each time step, agents submit a \emph{joint action} to the system supervisor consisting of a sequence of elementary action units. These action units will be performed in order upon execution of the joint action. The basic action units are as follows:
	
	\begin{enumerate}
		\item Catch - the agent acquires an available resource to be assigned to its consumer. Caught resources are held until consumed or a \emph{release} action is taken.
		\item Release - the agent returns a held resource to the pool of available resources.
		\item Wait - the agent delays consumption of resources until the subsequent time step. Wait actions may only occur once per joint action, and only as the last action in the sequence.
	\end{enumerate}
	
	\newpage
	
	\section{Consumer}
	
		\begin{definition}
			A \emph{consumer}, $c$, is a triple $(k, \phi, t)$, where $k \in \mathbb{N}$, $\phi$ is a specification, and $t \in \mathbb{R}^{+}$. The consumer $C$ is the subset of consumer $c$ which defined such that:
			\[
			c = (\phi , \sigma_m)
			\]
			\[
			C = 2^{\phi}*[0,\sigma_m)
			\]
		\end{definition}
		
		\begin{definition}
			Let \emph{service} $v$ be the available resource can be provide which related with a subset of resource $R'$ and a Boolean variable $\beta$ can be defined such that:
			\[
			v := (R', \beta)
			\]
			\[
			V = 2^{R}*{\ss}
			\]
		\end{definition}
	
	\section{Requirements}
	
		\begin{definition}
			A \emph{requirement} is a function $\reqfunc{\rho}{X}$, where $X$ is a set and $\mathbb{B} = \{0,1\}$. The set $X$ is referred to as the required set of $\rho$.
		\end{definition}
		
		\begin{definition}
			A requirement $\rho$ is said to be \emph{separable} if and only if it may be written as $\rho(x) = \prod_{i \in I}{\rho_{i}(x)}$, where $\forall i \in I$, $\rho_{i}(x)$ is a requirement with required set $X$.
		\end{definition}
		
		\begin{definition}
			A \emph{requirement operator} is a mapping, $\mathbf{Req}: \mathbf{Set} \rightarrow \mathbf{Set}$, such that: \[\reqop{\rho}{X} := \setbuild{x}	{X}{\rho(x) = 1}.\]
		\end{definition}
		
		\begin{proposition}
			Given a requirement $\rho = \rho_{1}\cdot\rho_{2}$, where $\rho_{1}$ and $\rho_{2}$ are requirements with required set X, then: 
			\[
			\reqop{\rho}{X}\stackrel{~}{=}\reqop{\rho_{1}}{X}\cdot\reqop{\rho_{2}}{X}
			\]
		\end{proposition}
		
		\begin{remark}
			The binary operation $\cdot$ between two requirements is the same as the symbol, $\wedge$, used in Boolean
 			algebra to represent the join, $and$, between two Boolean statements. Likewise, $\cdot$ operating on two sets
 			is the intersection operation $\cap$. 
		\end{remark}
		
		\begin{proof}
			The proof of this proposition is very straightforward. Let $X$ be a set and $\rho$ be a requirement with
 			required set $X$. Then,
			\begin{eqnarray}
			\reqop{\rho}{X} & = & \setbuild{x}{X}{\rho(x) = 1}\\ 
			& = & \setbuild{x}{X}{\rho_{1}(x)\cdot\rho_{2}(x) = 1}\\ 
			& = & \setbuild{x}{X}{\rho_{1}(x) = 1 \mathbin{and} \rho_{2}(x) = 1}\\ 
			& = & \setbuild{x}{X}{\rho_{1}(x) = 1}\cdot\setbuild{x}{X}{\rho_{2}(x) = 1}\\
			& = & \reqop{\rho_{1}}{X}\cdot\reqop{\rho_{2}}{X}
			\end{eqnarray}
		\end{proof}
	
	\section{Specifications}
	
		\begin{definition}
			A set, $X$, is said to be inspectable if and only if there exists a function,
			$\func{\psi}{X}{\prod_{i \in I}{X_{i}}}$, where $X \ne X_{i}, \forall i \in I$. This function is referred to as an
			\emph{inspection function} of $X$.
		\end{definition}
		
		\begin{remark}
			The inspection function may also be expressed as, $\psi(x) = (\psi_{i}(x))_{i \in I}.$
		\end{remark}
		
		\begin{definition}
			A \emph{specification} is a requirement, $\reqfunc{\phi}{X}$, such that the following conditions
			hold:
			
			\begin{enumerate}
				\item The required set, $X$, is inspectable.
				
				\item There exists a requirement, $\reqfunc{\varphi}{\prod_{i \in I}{X_{i}}}$,
				such that $\rho = \varphi \circ \psi$, where $\psi$ is an inspection function of $X$,
				with $X \ne X_{i}, \forall i \in I$. The requirement, $\varphi$, is referred to as an
				\emph{acceptance requirement} of X.
			\end{enumerate}
			
		\end{definition}
		
		\begin{theorem}
			Given a separable specification, $\phi$, with an acceptance requirement that is mutually independent,
			$\varphi(x_i)_{i \in I}$, then following isomorphism holds:
			
			\[
			\reqop{\phi}{X} = \prod_{i \in I}\reqop{\varphi_{i}}{X_{i}}
			\]
		\end{theorem}

	
	\newpage
	
\part{Reinforcement Model}

	\section{Data Collection}

	The data collected on consumers is as follows:
	\begin{table}

	\end{table}

	\section{Consumer Classification}
	Using Bayesian hierarchical clustering, consumers are classified into a certain consumer type represented by a
    numeric label (e.g. 1, 2). After classification, the consumer then enters their specification parameters and
    consumer type into a policy advisor which then outputs policy parameters for a scheduling agent to use, for
    the duration of their search.
	
	\section{Policy Advisor}
	The policy advisor is the mechanism which consumers utilize to encode their specification into their
    representative scheduling agents. These agents then use the output given by the policy advisor corresponding
    to their class in order to parameterize their policy. The policy advisor is defined as follows.

    \begin{definition}
        The \emph{policy advisor} is function $\mathcal{P}: \Gamma \rightarrow \Theta$, where $\Gamma$ is the specification
		parameter space and $\Theta$ is the policy parameter space.
    \end{definition}

	\section{System Supervisor}
	
	\indent
	
	Agent decisions are submitted as joint actions to a system supervisor, which resolves collisions between catch requests so as to maximize the net gain of all agents in the system. The gain function will be analytically constructed to provide functionality in the short-term, but will be replaced with a trained version once the reinforcement model has converged sufficiently for effective resolution. The supervisor then initiates consumption of collected resources, excluding those held by an agent whose joint action concluded with a wait operation.
	
	The supervisor also tracks the progression of each specification through the system, and distributes reward to the respective agents according to a function of the collected data and potential consumer feedback. In the instance of the scalable cloud, reward will be computed upon completion of the job, at which point the total time spent in the system (the sum of wait time and run time) and any consumer feedback will be available and may be factored into the calculation. To maximize fairness, longer wait times would be permissible for jobs that placed larger demands on the system (in cores or core hours). Total time is the main factor to consider, as minimizing wait time will maximize system utilization, while reduced run time is a result of optimized resource selection. Both times would be normalized relative to expected values derived from the specification parameters and the state of the system.
	
	\newpage    

	\begin{notation}
		\notate{\gamma}{specification parameter}\\
		\notate{\Gamma}{specification parameter space}\\
		\notate{C}{a subset of consumers}\\
		\notate{v}{a service}\\
		\notate{V}{Set of services}\\
		\notate{\rho(X)}{requirement function}\\
		\notate{\reqop{\rho}{X}}{requirement operator with parameter set X}\\
        \notate{\wedge}{and (joint)}\\
        \notate{\cap}{intersection}\\
        \notate{\psi(X)}{inspection function}\\
        \notate{\phi}{acceptance requirement}\\
		\notate{a}{an arriving process}\\
		\notate{A,G}{assignment actions}\\
		\notate{g}{an assignment process}\\
		\notate{d}{a departure process}\\
		\notate{D}{a departure action}\\
        \notate{G}{a graph with vertices V, and Edges E}\\
        \notate{\psi(X)}{inspection function}\\
        \notate{\path{G}{(n)}}{a path in graph G of length n}\\
        \notate{\sqcup}{disjoint union}\\
        \notate{\cyclepath{G}(v)}{cycle in graph G from vertex v to vertex v}\\
        \notate{\mathcal{P}(\gamma)}{policy advisor}
     \end{notation}

	\newpage
	
\end{document}

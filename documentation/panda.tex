\documentclass{article}



\title{Scala Job Scheduler}
\author{Matthew Brulhardt, Neo Pei, Kiran Eiden, Xiaolong Huang, Sam Zurl}




\begin{document}
	\maketitle

	\begin{abstract}
		Hello Abstract!
	\end{abstract}

	\section{Introduction}
	% Questions:
	%   1.  What is the problem that this paper is trying to solve?
	% 	2.  What is the aim of this paper, with respect to the problem that has been
	%   	established?

	\section{Previous Work}
	% Questions:
	% 	1.  Where does the information of the prescribed method get inspired from?
	% 	2.  What work has been done concerning partially observable state spaces in
	%       reinforcement learning?


	\section{Policy Advisor Network and Decision Architecture (PANDA)}

	% Questions:
	%	1.  What does this architecture plan to do that others can not do?
	%   2.  How is this model simpler than others?
	%   3.  What benefits does this architecture plan to produce?


	\subsection{Overview}
	% General Model Diagram (picture)
	% Explanation of labeled parts of the diagram and all the constituents that
	% make up the model.

	\subsection{Model}

	% Introduction to the model and the rationale for choosing this model over
	% other models.
	%
	% Questions:
	% 	1.  Why use reinforcement learning over traditional methods for scheduling?
	%	2.  What are the overall benefits that this model hopes to bring to the table?

	\subsubsection{Policy Advisor}
	% Questions:
	%   1.  What is the role of the policy advisor?
	%   2.  How does this interact with the rest of the model?
	%   3.  Is there any benefit to in using this?
	%   4.  How long will it take to train this?

	\paragraph{Topology}
	% Questions:
	% 	1.  How does the hierarchy topology connect with the pre-training?
	% 	2.  How does this hierarchy topology help in the reduction of training time?

	% (Insert Hierarchy Image)

	% Questions:
	% 	1.  What do all the parts of the diagram represent?
	%   2.  Where is the direct connection between the hierarchy and reduced training time?

	\subsubsection{Agent Model}

	\paragraph{Partial Oberservation and Attention Mechanism}
	% Questions:
	% 	1.  How do the agents observe the space and render a decision based what they
	%       observed?
	%   2.  How does restricting their attention to a certain portion of the space help in %       their decision making?
	%   3.  How do the agents interact with one another in this environent?
	%   4.  What is the process for conflict resolution between agents?

	\paragraph{Catch-Release-Wait Mechanism (CRW)}
	% Question:
	%   1.  How do the mechanics of this mechanism work?
	%   2.  What is there role concerning the model as a whole?
	%   3.  What are the benefits that could come with this mechanism?

	\subsubsection{System Supervisor}
	% Question:
	%   1.  What is the role of this part of the system?

	\section{Training}

	\subsection{Pre-Training}
	% Question:
	% 	1.  What needs to be pre-trained and why?
	% 	2.  Are there any obvious benefits in doing this?
	% 	3.  How long is pre-training going to take?
	% 	4.  How do you plan on mitigating the set backs that could come with this?

	\subsection{Model Training}
	% Question:
	% 	1.  How exactly do you plan on training the model?
	% 	2.  How long is this training going to take?
	% 	3.  How do you plan on mitigating the known set backs?

	\section{Discussion}
	% Question:
	% 	1.  To sum it all up, what are the major advantages of using PANDA for job
	%       scheduling?
	% 	2.  What are the set backs on this approach to solving the problem?
	% 	3.  How do you plan on mitigating the known set backs?

	\paragraph{Hybrid Approach}
	% Question:
	% 	1.  How does this approach mitigate the set backs that have been previously
	%       mentioned?
	% 	2.  How long will this hybrid approach have to be used until the real system is
	%       ready to take over?

\end{document}